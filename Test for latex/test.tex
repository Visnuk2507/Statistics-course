\documentclass{article}

% Language setting
% Replace `english' with e.g. `spanish' to change the document language
\usepackage[english]{babel}

% Set page size and margins
% Replace `letterpaper' with `a4paper' for UK/EU standard size
\usepackage[letterpaper,top=2cm,bottom=2cm,left=3cm,right=3cm,marginparwidth=1.75cm]{geometry}

% Useful packages
\usepackage{amsmath}
\usepackage{graphicx}
\usepackage[colorlinks=true, allcolors=blue]{hyperref}
\usepackage{pythontex}

\title{Introduktion til klinisk praksis på hospital}
\author{\textbf{Rapport
Skrevet af:} Visnukaran Kirubakaran
\noindent
\\
\\
\textbf{Studienummer:} cdj104}
\usepackage[utf8]{inputenc}
\usepackage{array}
\usepackage{geometry}
\geometry{a4paper, margin=1in}

\begin{document}
\maketitle
\noindent
Helloo, and welcome to my first report in latex!dwdrfrfdee
\noindent
In the following, you will see my first import of pythoncode in a latex document:
\noindent
\begin{pycode}
import sys
import os
sys.path.append(os.path.abspath("C:/Statistics course"))
from python_code import add
result = add(100,90)
print(result)
\end{pycode}
\noindent
The result of adding the numbers using the pythoncode is: \py{result}

\end{document}